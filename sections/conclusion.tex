\section{Conclusion}

In this paper, we investigate the possibility of spoofing  face authentication systems with a single  photo by bypassing its 3D liveness detection. 
We find that modulated structured light scatter patterns can spoof the depth camera to create fake face depth information. Meanwhile, the 3D liveness detection is vulnerable to such fake depths and RGB adversarial examples. 
Based on it, we propose the \texttt{DepthFake} attack, which projects the modulated structured light scatter pattern on the adversarial photo of a legitimate user to spoof 3D liveness detection and thus the face authentication system.
Evaluation with three commercial face authentication systems (Tencent Cloud, Baidu Cloud, and 3DiVi) and one commercial access control device demonstrates the effectiveness of \texttt{DepthFake}  attacks in the real world. In addition to the face authentication system analyzed in this paper, our work can be extended to other systems equipped with structured-light-based depth cameras. Future directions include exploring the security of the full workflow of face authentication and the security of other depth systems.