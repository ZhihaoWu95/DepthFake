% !TEX root = ../main.tex
%-------------------------------------------------------------------------------
\begin{abstract}
	%-------------------------------------------------------------------------------
Face authentication has been widely used in unlocking devices, securing financial payments, and providing access control to critical infrastructures. 
However, such an authentication system is threatened by the photo replay attack where an attacker uses a 2D photo to bypass the authentication. To safeguard such face authentication systems, the 3D liveness detection techniques are exploited to distinguish a 2D photo by detecting its liveness.
In this paper, we seek to answer whether the 3D liveness detection is secure enough and discover a new attack surface against such 3D-liveness-detection-based face authentication.
% vulnerability of 3D face authentication systems and %investigate the possibility to spoof them with a 2D photo. 
We propose the \texttt{DepthFake} attack that can spoof a 3D face authentication using only a 2D photo.
%the first practical attack in the real world against commercial face authentication systems without using the 3D mask or dummy. 
To achieve this goal, \texttt{DepthFake} first estimates the 3D depth information of a target victim's face from his 2D photo. Then, \texttt{DepthFake} projects the craft-designed scatter patterns that modulate the face depth information, in order to empower the 2D photo with 3D authentication properties.
We address practical challenges such as depth estimation errors from 2D photos, depth images forgery based on structured light, and the alignment of the RGB image and depth images.
% We performed \texttt{DepthFake} against the commercial structured-light-based depth camera AstraPro and 
We validated the performance of \texttt{DepthFake} with 3 commercial face authentication systems (i.e., Tencent Cloud, Baidu Cloud, and 3DiVi) and one commercial access control device. 
The results demonstrate that \texttt{DepthFake} achieves an overall attack success rate of $78.81\%$ and RGB-D attack success rate of $58.75\%$ in the real world. 
\end{abstract}