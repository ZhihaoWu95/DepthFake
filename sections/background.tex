% !TEX root = ../main.tex
\section{Background}
\label{sec:background}
%-------------------------------------------------------------------------------
In this section, we first present the work-flow of the 3D face authentication system and then introduce the liveness detection in detail.

\subsection{3D Face Authentication}
Face authentication is a biological authentication scheme used to verify the identity of a user. To prevent a face authentication system from photo replay attacks, 3D face authentication is invented by adding the 3D liveness detection module. 

A standard 3D face authentication system typically employs a visible light camera and an infrared camera. When a user attempts to authenticate, RGB and depth images are captured. As shown in Fig~\ref{fas_workflow}, the 3D face authentication process has the following three steps:
%When a user attempts to access a device or system with 3D face authentication, the RGB as well as the depth images of the users are captured by the visible light camera and the infrared camera, respectively, for further authentication. A standard 3D face authentication system works in three steps as shown in Figure.~\ref{fas_workflow}.

\textbf{Step 1: Face Detection.} 
The face detection step detects and crops the face regions in the RGB and depth images. For RGB images, face detection algorithms such as multi-task convolutional neural network (MTCNN) \cite{zhang2016joint} are used to locate the face area in the RGB images and output the bounding-box coordinates. Based on the face area in a RGB image, the face area in a depth image is extracted by mapping the face bounding-box coordinates of a RGB image to that of the depth image.

\textbf{Step 2: Liveness Detection.} The liveness detection step analyzes the liveness properties, such as the edges, textures, Moiré patterns
 from a RGB image and the 3D depth information from a depth image. 
This step aims to defend against spoofing attacks such as photo replay attacks, video replays attack, and mask-wearing attacks~\cite{chakka2011competition,anjos2011counter,raghavendra2015presentation, bhattacharjee2018spoofing, nesli2013spoofing}.

\textbf{Step 3: Face Comparison.} After the liveness detection step, the face comparison step employs comparison algorithms such as FaceNet~\cite{schroff2015facenet} to verify if the user is the legitimate one. The comparison algorithms often extract a feature vector from the newly captured face images and calculate the similarity distance between it and the enrolled feature vector stored in the database. Noted that face comparison only relies on the RGB images because of their high resolution.

From the work-flow of the face authentication system, we find that the key to fool a 3D face authentication system is to bypass its liveness detection module. In the following, we introduce the liveness detection in detail.

\begin{figure}[pt]
	\centerline{\includegraphics[width = 0.5\textwidth]{figures/face_auth_workflow.pdf}}
	\vspace{-0.15in}
	\caption{A 3D face authentication scheme first detects faces in the captured RGB and depth images, then determines whether those faces are from real people, and finally verifies whether they are from legitimate users.}
	\label{fas_workflow}
	\vspace{-0.15in}
\end{figure}

\subsection{Liveness Detection}
Liveness detection determines whether the person to be authenticated is a live person or not. 
It aims to reject non-living objects that try to obfuscate the face authentication system, e.g., a printed photo.
Existing liveness detection methods include two categories: 
(1) Active liveness detection requires the user to perform a predefined action, e.g., blink or nod. This method needs multiple captured images to recognize a pre-defied action and therefore is commonly used in cloud-service-based face authentication due to its extra requirements on computing resources. 
(2) Passive liveness detection, on the contrary, only uses one-shot images to determine whether the user is alive. 
Since passive liveness detection is lightweight and requires little user interaction, it is a popular method used nowadays in smartphones, smart locks, access control devices, etc~\cite{chakraborty2014overview}. Thus, in this paper, we target the 3D liveness detection in a passive manner. 

\subsection{3D Passive Liveness Detection}

3D passive liveness detection utilizes both RGB and depth images, and it may use one of the following types of depth camera: (1) structured light camera which uses the infrared scatter pattern to encode depth information, (2) time-of-flight (ToF) camera which calculates the depth by recording the infrared light echo time, and (3) binocular stereo camera, which obtains depth information by matching different perspectives photos taken by two cameras.
Among them, the structured-light-based depth camera is the most widely used one for face authentication systems, because of its high resolution, insensitivity to visible light, and low cost. For instance, it is used by FaceID~\cite{han2007face,bud2018facing}, and Smartlock\cite{waseem2020face}. Since the structured-light-based depth cameras are reported to occupy  almost 25\% among all the depth camera market~\cite{3dcameramarket},
in this paper, we focus on the passive liveness detection systems using structured-light-based depth cameras. Nevertheless, the methodology of \texttt{DepthFake} can be applied to other liveness detection methods.

\begin{figure}[pt] 
	\centerline{\includegraphics[width = 0.48\textwidth]{figures/structured_light_camera.pdf}}
	\vspace{-0.1in}
	\caption{The structured light depth camera actively projects a constant scatter pattern to the object, e.g., a face, then captures the reflected scatter pattern and calculates the depth of the target by measuring its displacement from the template scatter pattern for each scatter point. }
	\label{depth_camera}
	\vspace{-0.15in}
\end{figure}


%A structured-light-based 3D passive liveness detection module (3D liveness detection module in short) consists of both hardware and software components.
%, where the former captures the face information of the user pending authentication while the latter processes the information and makes decisions.
%\textbf{Hardware.}
%The hardware component  of a 3D liveness detection module include a visible camera and a structured light depth camera, which capture the RGB and depth images of a legitimate user  simultaneously~\cite{zanuttigh2016time, daneshmand20183d}. 
\textbf{Structured-light-based 3D liveness detection.} A standard structured light depth camera contains a scatter projector and an infrared camera, as shown in Fig.~\ref{depth_camera}. Most scatter projector projects a constant infrared pattern containing tens of thousands of scatter points. During calibration in factory, the infrared camera captures the scatter patten at the reference depth plane and stores it as a template scatter pattern. To obtain a depth image of a user,  the infrared camera captures the reflected scatter pattern, which is distorted due to the different depths of the face, and it calculates the depth information by measuring the displacements between each scatter point with the one stored in the template scatter pattern.


%
%When capturing a depth image of a target user, the reflected scatter pattern will be distorted due to the different depths of the face. Then, the infrared camera will capture the distorted scatter pattern and calculate the depth information by measuring the displacements between each scatter point with the one stored in the template scatter pattern.

To illustrate how the structured light depth camera calculates the depth information from scatter point displacements, we consider a single scatter point.
% For simplicity, we take the depth calculation for a single scatter point as an example to illustrate how the structured light camera measures the depth from scatter displacements. 
As shown in Fig.~\ref{imaging_mechanism}, as an infrared beam of $x_p$ is reflected at the plane of reference depth $d_{ref}$ and tatget depth $d_t$, the scatter points on the image sensor of infrared camera can be denoted as $x_{ref}$ and $x_{t}$, respectively. 
%a scatter point $x_p$ reflected by the planes of  reference depth $d_{ref}$ and tatget depth $d_t$ are denoted as $x_{ref}$ and $x_{t}$ on the imaging plane of the infrared camera.
Based on the rule of the similar triangle, we can calculate $d_t$ as follows:
\begin{equation}
	d_t= \frac{d_{ref}Lf_c}{Lf_c - d_{ref}\Delta x_c}
	\label{d_cal}
\end{equation}
where $\Delta x_c=x_{ref}-x_{t}$ is the displacement between $x_{ref}$ and $x_{t}$. The baseline length $L$, the focal length $f_c$, and reference depth $d_{ref}$ are constant and known to the depth camera. Thus,  the camera can obtain the depth of a single point and multiple points across the entire face.

%Based on the similar triangles rule, we can calculate $x_{ref}$ and $x_{t}$ as follows:
%
%\begin{equation}
%	x_{ref}=\frac{f_c}{d_ref}(L-\frac{x_pd_ref}{f_p}) 
%	\label{xc1_cal}
%\end{equation}
%\begin{equation}
%	x_{t}=\frac{f_c}{d_t}(L-\frac{x_pd_t}{f_p}) 
%	\label{xc2_cal}
%\end{equation}
%where $f_c$ and $f_p$ are the focal length of the infrared camera and projector, and  $L$  is the baseline length (distance) between infrared camera and projector. 
%Based on them, we can calculate ${d_t}$ as follows:
%\begin{equation}
%	%\frac{1}{d_1} - \frac{1}{d_2} = \frac{\Delta x}{Lf_c} 
%	d_t= \frac{d_refLf_c}{Lf_c - d_ref\Delta x_c}
%	\label{d_cal}
%\end{equation}
%where $\Delta x_c=x_{ref}-x_{t}$ is the displacement between $x_{ref}$ and $x_{t}$. The baseline length $L$, the focal length $f_c$, and reference depth $d_ref$ are constant and known to the depth camera. In such way, the camera can obtain the depth of a single point and even a complete face.


\begin{figure}[!t]
	\centering
	\includegraphics[width=0.48\textwidth]{figures/imaging_mechanism.pdf} 
	\vspace{-0.1in}
	\caption{Imaging mechanism of the structured light camera. The scatter projector projects the scatter point to the reference depth plane and the target depth, then the infrared camera captures the reflected scatter points and calculates the target depth by measuring the displacement between them. }
	\label{imaging_mechanism}
	\vspace{-0.15in}
\end{figure}

After obtaining RGB and depth images, the 3D liveness detection module utilizes both of them to determine the liveness.
For an RGB image, deep learning algorithms such as Convolution Neural Networks (CNNs)~\cite{yang2014learn, chen2020face, luo2018face} or Vision Transformer (Vit)~\cite{george2020effectiveness}, are used to extract features, e.g., edges, textures, Moiré patterns or features in the frequency domain in the RGB image, to determine whether the RGB stands for a real person.
For a depth image, the region with face is extracted by mapping to the RGB image, and then feature point matching algorithms~\cite{goswami2014rgb} or CNNs~\cite{george2021cross, te2020exploring} are used to determine whether the face belongs to a real person by detecting the special geometric structure of the face. 

%\textbf{Hardware.} The
%3D liveness detection module uses a visible camera to capture the RGB image of a legitimate user and a structured-light depth camera~\cite{zanuttigh2016time, daneshmand20183d} to capture the depth image at the same time. 
%A standard structured-light depth camera contains a scatter projector and an infrared camera, as shown in Fig.~\ref{depth_camera}.
%During capturing depth images,  the projector first projects a fixed infrared scatter pattern to the target, which will suffer from different deformations due to the different depths of the target. Then, the infrared camera will capture the reflected scatter pattern from the object and compare it with the projected scatter pattern.  By measuring the displacement of each scatter with the disparity estimation algorithm~\cite{fanello2016hyperdepth}, it calculates the depth of the target.

%\textbf{Software.} After obtaining RGB and depth images at the same time, 3D liveness detection module utilizes both of them for decisions.
%For RGB images, it uses deep learning algorithms, e.g., Convolution Neural Networks (CNNs)~\cite{yang2014learn, chen2020face, luo2018face} or Vision Transformer (ViT)~\cite{george2020effectiveness}, to extract features such as edges, textures, moiré patterns or features in the frequency domain,  and then utilizes a binary classifier to determine them as true or fake faces with confidence scores.
%For depth images, it first maps it with the RGB image and locates the region of the face. Then, it uses feature point matching~\cite{goswami2014rgb} or CNNs~\cite{george2021cross, te2020exploring} to determine whether the face belongs to a real person by detecting the special geometric structure of the human face. 




%\textbf{Hardware. }
%3D passive liveness detection uses a visible camera to capture the RGB image of a legitimate user and a structured-light depth camera~\cite{zanuttigh2016time, daneshmand20183d} to capture the depth image at the same time. 
%A standard structured-light depth camera contains a scatter projector and an infrared camera, as shown in Fig.~\ref{depth_camera}.
%During capturing depth images,  the projector first projects a fixed infrared scatter pattern to the target, which will suffer from different deformations due to the different depths of the target. Then, the infrared camera will capture the reflected scatter pattern from the object and compare it with the projected scatter pattern.  By measuring the displacement of each scatter with the disparity estimation algorithm~\cite{fanello2016hyperdepth}, it calculates the depth of the target.
%
%\textbf{Software. } After obtaining RGB and depth images at the same time, 3D liveness detection utilizes both of them for decisions.
%For RGB images, it uses deep learning algorithms, e.g., Convolution Neural Networks (CNNs)~\cite{yang2014learn, chen2020face, luo2018face} or Vision Transformer (ViT)~\cite{george2020effectiveness}, to extract features such as edges, textures, moiré patterns or features in the frequency domain,  and then utilizes a binary classifier to determine them as true or fake faces with confidence scores.
%For depth images, it first maps it with the RGB image and locates the region of the face. Then, it uses feature point matching~\cite{goswami2014rgb} or CNNs~\cite{george2021cross, te2020exploring} to determine whether the face belongs to a real person by detecting the special geometric structure of the human face. 

%(2) IR-based liveness detection, which actively projects infrared light to target object and identifies the real face through comparing the difference of reflection features between the real human face and other spoofing materials. 
%(3) RGB+IR liveness detection, which combines the advantages of RGB-based and IR-based liveness detection and can work in different lighting conditions. 
%3D passive liveness detection is RGB-Depth-based (RGBD-based in short), which uses the depth information as an auxiliary feature in anti-spoofing. The algorithm reconstructs a 3D structure from depth image and uses it to determine whether the target is a real person or not.

%\subsection{Spoofing Attacks against Face Authentication}
%
%Existing spoofing attacks against face authentication mainly focus on the face comparison step, i.e., approximating the feature vector of a legitimate user in the face feature representation space. In this area, much work~\cite{sharif2016accessorize, singh2022powerful, nguyen2020adversarial, guo2021meaningful, komkov2021advhat, zolfi2021adversarial} has demonstrated the feasibility of fooling deep-learning-based face comparison algorithms with adversarial examples, which can render the algorithms identify an adversary as a legitimate user.However, those attacks mainly work for RGB images and may lose their attack capabilities when depth features are used \cite{belli2022personalized}.
%Another aspect of spoofing face authentication is to deceive its 3D liveness detection. Existing work to achieve it is using an ultra-realistic 3D face mask~\cite{bhattacharjee2018spoofing}, but it requires a detailed 3D modeling of the victim and a high cost (over $\$4,000$).In this paper, we investigate the possibility of fooling face authentication with a low-cost printed photo.





%Face spoofing attacks use artifacts to spoof liveness detection systems. The most common way is the replay attack. The current replay attacks usually use high-precision printer to print the photo or high-resolution monitor to replay the video of the victim. Additionally, there are attackers who design ultra-realistic face mask to spoof liveness detection. However, the multi-modal liveness detection can prevent these attacks by identifying the specific texture or optical features.  
%
%There are other face spoofing attacks are emerging nowadays which taking over the camera feed and injecting the pre-designed image or video. Deepfake attack is one of the injecting attack, which uses synthetic face data and injects it into the camera. However, injecting attacks are all need intrusion into the hardware or software of liveness detection systems and are difficult to deploy in the real world.
%
%Therefore, in this paper, we conduct replay attacks which are more easier to deploy in the real world and propose different replay attacks for different modalities to against multi-modal liveness detection systems.