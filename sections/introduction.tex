% !TEX root = ../main.tex
\section{Introduction}
%-------------------------------------------------------------------------------
A face authentication system verifies the legitimacy of a user by matching a human face, usually in the form of a digital image, against the ones in the database. Such a system has been used in unlocking devices, securing financial payments, and physical access control to critical infrastructures. Because face authentication requires no physical contact compared to other biometrics such as fingerprints, it has become one of the most popular authentication methods, especially in the COVID-19 era. The face authentication systems that purely rely on 2D images are known to be vulnerable to 2D replay attacks~\cite{chakka2011competition,anjos2011counter,raghavendra2015presentation}, e.g., an attacker may fool the systems with a printed photo of the legitimate user.
To cope with such attacks, manufacturers such as Apple, Baidu, Tencent, etc.,~\cite{faceid, baidu, tencent} start to utilize 3D liveness detection techniques to distinguish a real human from a printed photo, and we call such authentication systems \textit{3D face authentication systems}. Specifically, they exploit a depth camera to obtain the RGB and depth images to extract 3D properties of human faces, e.g., the depth and the skin texture of faces. Since a printed 2D image does not contain 3D properties, 3D face authentication systems can block existing 2D spoofing attacks~\cite{souza2018far, marcel2014handbook}.



%To prevent such replay attacks, the 3D liveness techniques are exploited to secure the authentication process, and to distinguish the replayed 2D photo by checking its 3D liveness property. Recently, manufacturers such as Apple, Baidu, Tencent, etc.,~\cite{faceid, baidu, tencent} utilize and implement such 3D liveness detection. 
%Specifically, it uses a depth camera to obtain the RGB and depth images as the 3D information to determine whether the users are truly present. Thus, the 2D replay attack is impossible to break the 3D face authentication system since the 2D photo lacks the required 3D features, and existing spoofing attacks~\cite{souza2018far, marcel2014handbook} indeed fall short.

\begin{figure}[t]
	\centerline{\includegraphics[width = 0.48\textwidth]{figures/intro.pdf}}
	\vspace{-0.1in}
	\caption{The \texttt{DepthFake} attack bypasses the 3D face authentication with liveness detection by projecting a crafted structured light scatter pattern onto a public 2D photo of the target victim.}
	\label{intro}
	\vspace{-0.2in}
\end{figure}

In this paper, we seek to investigate \emph{``Is it indeed impossible to spoof a 3D face authentication system using only one single 2D photo?''} Theoretically, if one can forge the depth information and play it to empower the 2D photo with 3D properties, such that the obtained depth images and RGB images match the ones of the legitimate users, and the 3D face authentication can be fooled. %will obtain to empower the 2D photo with 3D properties. 
We call such an attack \alias and envision the following scenario, as shown in Fig.~\ref{intro}. An attacker can obtain an RGB photo of a victim and estimate the 3D properties of the victim beforehand. To impersonate the victim, she emits an infrared pattern with the victim's 3D depth information embedded and perfectly aligned with the 2D photo, such that it appears as if the victim is standing in front and thus bypass the 3D liveness detection.

  %The \texttt{Depthfake} attack first estimates the depth information of the victim from his public 2D photo, then spoofs the depth camera by emitting the infrared pattern which contains the depth information. At last, by aligning the 2D photo with the forgery depth, the \texttt{Depthfake} attack can successfully forge the 3D information of the victim and thus bypass the 3D liveness detection.

%The answer is Yes. We find the key to bypassing 3D liveness detection is to forge the depth information and replay it to empower the 2D photo with 3D properties. To this end, we propose the \texttt{DepthFake} attack as shown in Fig.~\ref{intro}. The \texttt{Depthfake} attack first estimates the depth information of the victim from his public 2D photo, then spoofs the depth camera by emitting the infrared pattern which contains the depth information. At last, by aligning the 2D photo with the forgery depth, the \texttt{Depthfake} attack can successfully forge the 3D information of the victim and thus bypass the 3D liveness detection.

\alias is made challenging because of the constraints of one photo and the limited information of the target authentication system. For instance, the authentication algorithm parameters and details are unknown, and it is unclear how the depth-based and the RGB-based liveness detection are performed. Nevertheless, the key of a successful \alias is to answer the following questions. 
%\begin{enumerate}[noitemsep]
%However, the design and implementation of \texttt{DepthFake} have several challenges in the real world  against the black-box settings on commercial face authentication systems:
%\end{enumerate}
\begin{inparaenum}[\itshape a\upshape)]
\item How to obtain the depth information from one single photo?
\item How to convert the depth information into an infrared pattern such that it can spoof the target depth camera?
\item How to modify the printed photo such that it can fool RGB-based liveness detection simultaneously?
\end{inparaenum}

To overcome the above challenges, we first propose a CNN-based deep learning model with the weighted loss function to estimate the depth information out of a single photo.
Then, we model the depth measurement process and design the mapping function that can convert the digital depth image into the desired scatter pattern such that the captured depth images appear similar to the ones of a live human. 
Finally, we propose an evolutionary-based RGB adversarial attack method with color calibration, and a face region alignment scheme to form an RGB-D attack, i.e., bypass both RGB-based and depth-based liveness detections.
We validate the effectiveness of \texttt{DepthFake} attacks on three commercial face authentication systems, e.g., Tencent Cloud, Baidu Cloud, and 3DiVi, and a commercial access control device in a laboratory setup. The validation on  \textcolor{red}{50} users shows that \texttt{DepthFake} achieve a success rate of  \textcolor{red}{$79.4\%$} for the systems with depth-based liveness detection and a success rate of  \textcolor{red}{$59.4\%$} for the systems with both depth-based and RGB-based liveness detection. In summary, our contributions include the points below:
\begin{itemize}	
	\item We identify the vulnerabilities in the 3D liveness detection of the face authentication system, and propose to spoof a 3D  face authentication system using a single 2D photo.	
	\item We design \texttt{DepthFake} that can spoof commercial face authentication systems by projecting the carefully-crafted structured light scatter pattern on a printed photo of a victim.	
	\item We validate \texttt{DepthFake} on a commercial structured light depth camera (i.e., Astra Pro), three commercial face authentication systems (i.e., Baidu Cloud, Tencent Cloud, 3DiVi), and one commercial access control device in the laboratory set up, and achieve a success rate of  \textcolor{red}{$79.4\%$} for the systems with depth-based liveness detection and a success rate of  \textcolor{red}{$59.4\%$} for the systems with both depth-based and RGB-based liveness detection.
\end{itemize}

\texttt{DepthFake} attacks server as the first attempt to fool 3D face authentication systems. To enhance the security of existing systems, we recommend two defense methods:
\begin{inparaenum}[\itshape a\upshape)]
\item Randomizing scatter patterns to make it extremely difficult for attackers to forge depth information, while existing depth cameras use a fixed scatter pattern.
\item  Enhancing 3D liveness detection models to detect the forgery depth information and adversarial examples.
\end{inparaenum}
% First, the single 2D photo captured from the victim's social media do not contain any depth information thus we need to estimate the depth image from it. Second, the estimated depth image shall be modulated into the structured light scatter pattern to spoof the depth camera for generating the forgery depth image. Third, launching attacks both in RGB and depth images to ensure the effectiveness under RGB-D liveness deteciton mode. 


%It seems impossible that a single 2D photo cannot hold the 3D features required to pass the 3D liveness detection and existing spoofing attacks~\cite{souza2018far, marcel2014handbook} indeed fall short. Essentially, we find the key to bypass the 3D liveness detection is to forge the depth information.  
%To this end, we try to use only a single 2D photo to break the 3D face authentication system, and propose the \texttt{DepthFake} attack. 
%
%The \texttt{Depthfake} attack first estimates the depth information of the victim from his public 2D photo, then forges the depth information in the real world by spoofing the structured light depth camera. In this way, a 2D photo is empowered with depth information. In addition, to enhance the attack performance against RGB-D-based liveness detection (i.e., the form of 3D livenss detection), we resort to adversarial example attacks to achieve uniformed spoofing.		


% find that a face authentication system includes a face detection phase, a liveness detection phase, and a face comparison phase, and the three phases work in serial. To bypass the 3D liveness detection, the key is to forge the depth information.

% propose the \texttt{DepthFake} attack, which uses only a single 2D photo to break the 3D face authentication system as shown in Fig.~\ref{intro}. 

%the face authentication systems usually use the RGB-D liveness detection for security purposes, the \texttt{Depthfake} attacks need to be facilitated with the capability of bypassing RGB-based liveness detection such as the edge detection or moiré-pattern check~\cite{yang2014learn, chen2020face, luo2018face}, we resort to the adversarial example attacks and add patterns such as white squares to the printed photos to achieve a flexible spoofing.	
%\texttt{DepthFake} attacks first generate the desired structured-light scatter pattern (normally IR) by measuring the displacements between the reflected and the legitimate ones. Then, the generated structured-light scatter pattern is projected onto the printed photo of the target victim to forge the depth information. 
%In this way, a 2D photo is empowered with the depth information. Our preliminary analysis in Sec.~\ref{sec:preliminary} shows that naively replaying a structured light scatter pattern can generate the depth information but has difficulty in deriving a desired one. In addition, to facilitate \texttt{DepthFake}  attacks with the capability of bypassing RGB-based liveness detection such as the edge detection or moiré-pattern check~\cite{yang2014learn, chen2020face, luo2018face}, we resort to the adversarial example attacks and add patterns such as white squares to the printed photos to achieve a flexible spoofing.	

%First, unlike traditional multi-view depth estimation methods, we obtain the depth information from a single public photo of the victim, which may introduce potential biases.
%Second, the structured light depth camera measures depth information using a specific scatter pattern, thus the depth images shall be converted to the desired scatter pattern.
%Third, the RGB adversarial examples may suffer from the colors shifted and need to align with depth images against the RGB-D liveness detection.

%To prevent 2D replay attacks, manufacturers such as Apple and Microsoft~\cite{faceid} utilize 3D liveness detection techniques to secure the authentication process, which is known as the 3D face authentication system. Specifically, it utilizes a depth camera to obtain the 3D information and identify whether the real user truly present.

% A face authentication system is a technology capable of matching a human face, usually in the form of a digital image, against a database of faces, to authenticate the legitimacy of the user. The enabled services include but are not limited to unlocking devices, securing financial payments, and physical access control to critical infrastructures. Especially, face authentication is regarded as a secure and common authentication method compared to other biometric solutions such as fingerprint authentication due to its non-contact property during the COVID-19 era.

%However, a face authentication system has been shown to be vulnerable to spoofing attacks.
%A naive but efficient idea is to use a printed photo of the legitimate user, which is also known as the 2D replay attack~\cite{chakka2011competition,anjos2011counter,raghavendra2015presentation}. To prevent such 2D replay attacks, face authentication systems nowadays are facilitated with liveness detection techniques. 
%Liveness detection requires users to participate in the authentication process, e.g., shaking their heads, or obtaining their 3D information to ensure they are truly present. In particular, due to the security and user-friendly consideration, the 3D liveness detection is being adopted by more manufacturers, such as Apple, Microsoft, etc. ~\cite{faceid}.